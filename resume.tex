\documentclass[letterpaper,11pt]{article}

\usepackage[noTeX]{mmap}
\pdfinclusioncopyfonts=1
% \usepackage[noTeX]{mmap}
\usepackage[utf8]{inputenc}
\usepackage[T1]{fontenc}
\usepackage[defaultsans]{lato}
\renewcommand{\familydefault}{\sfdefault}
% \usepackage{lato}
% \usepackage[rm]{roboto}

% \usepackage{latexsym}
\usepackage[empty]{fullpage}
\usepackage{titlesec}
% \usepackage{marvosym}
\usepackage[usenames,dvipsnames]{color}
% \usepackage{verbatim}
\usepackage{enumitem}
\usepackage[pdftex]{hyperref}
\usepackage{fancyhdr}
% \usepackage[sfdefault]{roboto}
\usepackage{xcolor}
% \usepackage{fontspec}

% Settings for better ATS parsing
\usepackage{microtype}
\DisableLigatures{encoding = *, family = *}

\input{glyphtounicode}
\pdfgentounicode=1

%%%

\pagestyle{fancy}
\fancyhf{} % clear all header and footer fields
\fancyfoot{}
\renewcommand{\headrulewidth}{0pt}
\renewcommand{\footrulewidth}{0pt}

% Adjust margins
\addtolength{\oddsidemargin}{-0.375in}
\addtolength{\evensidemargin}{-0.375in}
\addtolength{\textwidth}{1in}
\addtolength{\topmargin}{-.5in}
\addtolength{\textheight}{1.0in}

\urlstyle{same}

\raggedbottom
\raggedright
\setlength{\tabcolsep}{0in}

% Sections formatting
\titleformat{\section}{
  \vspace{-4pt}\scshape\raggedright\large
}{}{0em}{}[\color{black}\titlerule \vspace{-5pt}]

\definecolor{CoolBlue}{HTML}{0589f9}

\hypersetup{
  colorlinks,
  linkcolor=CoolBlue,
  urlcolor=CoolBlue,
}

%-------------------------
% Custom commands
\newcommand{\resumeItem}[2]{
  \item\small{
    \textbf{#1}{: #2 \vspace{-2pt}}
  }
}

\newcommand{\resumeSubheading}[4]{
  \vspace{-1pt}\item
    \begin{tabular*}{0.97\textwidth}{l@{\extracolsep{\fill}}r}
      \textbf{#1} & #2 \\
      {\small#3} & {\small #4} \\
    \end{tabular*}\vspace{-5pt}
}

\newcommand{\resumeSubItem}[2]{\resumeItem{#1}{#2}\vspace{-4pt}}

\renewcommand{\labelitemii}{$\circ$}

\newcommand{\resumeSubHeadingListStart}{\begin{itemize}[leftmargin=*]}
\newcommand{\resumeSubHeadingListEnd}{\end{itemize}}
\newcommand{\resumeItemListStart}{\begin{itemize}}
\newcommand{\resumeItemListEnd}{\end{itemize}\vspace{-5pt}}

\newcommand{\startList}{\begin{itemize} \small}
\newcommand{\finishList}{\end{itemize} \vspace{-5pt}}

%-------------------------------------------
%%%%%%  CV STARTS HERE  %%%%%%%%%%%%%%%%%%%%%%%%%%%%


\begin{document}

%----------HEADING-----------------
\begin{tabular*}{\textwidth}{l@{\extracolsep{\fill}}r}
  \textbf{\Large Semen Panenkov} & Email : \href{mailto:smn.pankv@gmail.com}{smn.pankv@gmail.com}\\
  \href{https://smntin.github.io}{https://smntin.github.io} & Mobile : +49 176 84406540 \\
\end{tabular*}


% \section{Profile}
%   Passionate student doing programming for own amusement since early childhood. Constantly learning, researching new technologies, and gaining new skills. Seeking to use the right tool in the right place. Focusing on the code quality and efficiency while keeping up with the deadlines.

%-----------EXPERIENCE-----------------
\section{Work experience}
  \resumeSubHeadingListStart
    \resumeSubheading
      {JetBrains}{Remote, Germany}
      {Research Intern at Programming Languages and Tools Lab}{Dec. 2022 -- Jun. 2023}
      \startList
        \item Developed the first formal specification of the Graph Query Language (GQL) in Coq.
        \item Implemented and proved correctness of the query evaluation via execution plans for two reference databases: Neo4j and RedisGraph.
        % Published the first framework for formal verification of query executions via <итд>
        \item Architected and planned the whole project, which turned into 6k lines of proofs.
        \item Employed \href{https://github.com/cyphercert/opencypher-coq}{this project} as a bachelor's thesis, achieving a score of 94/100 and, subsequently, presenting this project at the annual graph databases community meeting in Seattle, US.
      \finishList
    \resumeSubheading
      {Saint Petersburg State University}{Saint Petersburg, Russia}
      {OCaml Developer}{Feb. 2022 -- Jun. 2022} %TODO - норм даты
      \startList
        \item Added user-defined algebraic data types and pattern-matching to \href{https://github.com/SmnTin/mtt-lang/tree/ADT}{an interpreted functional language}, enabling the practical implementation of realistic programs.
        \item Refactored the language syntax and the type system, making the language more pleasant to use.
      \finishList
    \resumeSubheading
      {Freelance}{Saint Petersburg, Russia}
      {C++ Developer}{Jul. 2019 -- Sep. 2019}
      \startList
        \item Developed \href{https://github.com/SmnTin/PathBuilder}{a web service} for constructing efficient custom travel routes through specified locations, respecting the working hours and public transit schedule, which became the main feature of a tourist trip planning mobile app.
        \item Tightly collaborated with the customer, which resulted in reworking the original algorithm to meet the efficiency requirements by increasing the maximum number of processed locations from 10 to hundreds.
        
      \finishList
  \resumeSubHeadingListEnd

%-----------PROJECTS-----------------
\section{Selected projects}
  \resumeSubHeadingListStart
    \resumeSubItem{\href{https://github.com/SmnTin/lincheck}{Lincheck}}
      {A \textbf{Rust} library for testing concurrent data structures for adherence to a sequential specification, built on top of a concurrency model-checker and property-based testing framework.}
    \resumeSubItem{\href{https://github.com/bot-mne-v-rot/high-shift-engine}{High-Shift Engine}}
      {A data-oriented game engine in \textbf{C++20} designed for maintaining large changing structures of entities. Leveraged advanced type system features to provide the most flexible API.}
    \resumeSubItem{\href{https://github.com/SmnTin/bachero}{Bachero}}
      {A 2d game engine based on SDL2 in \textbf{C++17} with a custom heavily-optimized physics engine, handling thousands of moving objects while keeping the frame rate above 60.}
    \resumeSubItem{\href{https://github.com/SmnTin/huffman-archiver}{Huffman Archiver}}
      {An archiver with canonical Huffman codes in \textbf{C++17} designed for fast and memory-efficient encoding and decoding of big files. Achieved throughput is about 300 MB/s on x86.}
    \resumeSubItem{\href{https://github.com/SmnTin/AIst}{AIst}}
      {The embedded software in \textbf{C++11} for an autonomous robot based on NVidia Jetson 2, controlled by a stack-based state machine, which supports recognition of the lane, traffic signs and traffic lights.}
    \resumeSubItem{\href{https://github.com/SmnTin/simple-type-checker/tree/system-f}{System F}}
      {A type checker in \textbf{Haskell} for Church-style Polymorphic Lambda Calculus (System F). Covered parsing, type checking and pretty-printing with 120 unit tests.}
    % \resumeSubItem{\href{https://github.com/SmnTin/prog-2020-text-index-SmnTin}{Text Index}}
    %   {A tool in \textbf{Kotlin}, built with clean architecture principles in mind, that provides fast text indexing and querying.}
  \resumeSubHeadingListEnd

%-----------EDUCATION-----------------
\section{Education}
  \resumeSubHeadingListStart
    \resumeSubheading
      {Jacobs University Bremen}{Bremen, Germany}
      {Bachelor of Computer Science \textbf{GPA:} B+}{Sep. 2022 -- Jun. 2023}
      \startList
        \item \textbf{Courses}: Distributed Systems, Semantics of Programming Languages, Dependent Types, Game Theory
      \finishList
    \resumeSubheading
      {Saint Petersburg State University}{Saint Petersburg, Russia}
      {Bachelor of Computer Science and Mathematics \textbf{GPA:} B+}{Sep. 2020 -- Aug. 2022}
      \startList
        \item \textbf{Courses}: Computer Architecture, Operating Systems, Algorithms and Data Structures, Linear Algebra, Discrete Mathematics, Databases, Theoretical Informatics, Mathematical Analysis, Probability Theory
      \finishList
  \resumeSubHeadingListEnd

%-----------ACADEMIC ACHIEVEMENTS-----------------
\section{Academic achievements}
  \resumeSubHeadingListStart
    % \resumeSubheading{JacobsHack Hackathon}{}
    % {1/100 absolute winner with a team of 4 people}{2023}
    \resumeSubheading{All-Russian School Olympiad in Informatics}{}
    {Top 200 prize winner in the finals out of millions participants}{May 2020}
    \resumeSubheading{Innopolis University Olympiad in Robotics}{}
    {1/20 absolute winner in the Intelligent Autonomous Cars category.}{Jun. 2019}
  \resumeSubHeadingListEnd

% %--------EXTRACURRICULAR ACTIVITIES------------
% \section{Extracurricular Activities}
%   \resumeSubHeadingListStart
%     \resumeSubheading{Student council member}{SPbSU}
%       {Organizing coding events at the faculty.}{Feb. 2021 -- Feb. 2023}
%   \resumeSubHeadingListEnd

% %-------------------------------------------

\end{document}